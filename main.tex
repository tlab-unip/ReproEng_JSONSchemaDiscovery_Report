
% VLDB template version of 2020-08-03 enhances the ACM template, version 1.7.0:
% https://www.acm.org/publications/proceedings-template
% The ACM Latex guide provides further information about the ACM template

\documentclass[sigconf, nonacm]{acmart}

%% The following content must be adapted for the final version
% paper-specific
\newcommand\vldbdoi{XX.XX/XXX.XX}
\newcommand\vldbpages{XXX-XXX}
% issue-specific
\newcommand\vldbvolume{14}
\newcommand\vldbissue{1}
\newcommand\vldbyear{2020}
% should be fine as it is
\newcommand\vldbauthors{\authors}
\newcommand\vldbtitle{\shorttitle} 
% leave empty if no availability url should be set
\newcommand\repengpackageurl{https://github.com/tlab-unip/ReproEng_JSONSchemaDiscovery}
\newcommand\originalprojecturl{https://github.com/gbd-ufsc/JSONSchemaDiscovery}
% whether page numbers should be shown or not, use 'plain' for review versions, 'empty' for camera ready
\newcommand\vldbpagestyle{plain} 

\begin{document}
\title{RepEng Project: A Replication Package for JSONSchemaDiscovery}

%%
%% The "author" command and its associated commands are used to define the authors and their affiliations.

\author{Qiang Tang}
\affiliation{%
	\institution{University of Passau}
	\city{Passau}
	\country{Germany}
}
\email{tang08@ads.uni-passau.de}

\maketitle

%%% do not modify the following VLDB block %%
%%% VLDB block start %%%
\ifdefempty{\vldbavailabilityurl}{}{
	\vspace{.3cm}
	\begingroup\small\noindent\raggedright\textbf{Artifact Availability:}\\
	The source code, data, and/or other artifacts have been made available at \url{\repengpackageurl} \newline \url{\originalprojecturl}.
	\endgroup
}
%%% VLDB block end %%%

\section{Introduction}

In 2018, JSONSchemaDiscovery\cite{jsonschema} introduced a method for extracting schemas from JSON or extended JSON documents stored in a NoSQL document-oriented database or other document repository. The article posits that the introduced extraction method may be as accurate, equivalent or superior to related work.

Undoubtedly, this research has provided us an excellent tool for JSON Schema extraction. Nevertheless, due to limitations in previous scientific tools, the structure of the original project presents a significant challenge for both researchers and users aiming to replicate the research results. Therefore, we have provided a replication package for JSONSchemaDiscovery, enabling people to run the tool effortlessly on various setups and execute the evaluation steps necessay to replicate the results.

\section{Package Overview}

\subsection{Challenges in Replication}

There are several challenges when attempting to replicate the research. Firstly, the original research has provided us a web application to facilitate extraction operations. However, as the application was built with node.js years ago, many dependencies have become deprecated. When attempting to build the application without any modification, it typically results in an error. Secondly, the datasets required for evaluation are difficult to find, and some of them need preprocessing before they can be used. Additionally, due to the dependency on a database, it is necessary to connect the web app with the database when launching the app, and data also needs to be loaded into the database before performing the evaluation.

\subsection{Package Components}

To address these challenges, the replication package includes Docker recipes and scripts for automated evaluation. Docker is used to construct and operate the web application provided by the research. Furthermore, to streamline server deployment efforts, a docker-compose file is included to connect the MongoDB server with the web application. Both the data required for the web application and test datasets will be stored in the database.

\subsection{Steps to Replicate}

In order to replicate the research results, one simply needs to execute the scripts within the container after launching the servers using docker compose. The scripts contain a series of operations designed to ensure a reproducible evaluation process. Initially, the datasets will be loaded to initialize the database. Subsequently, extraction tasks will be assigned through a series of requests to the web server. Following this, a schema and related metadata will be output for examination. This output can then be compared with the ground truth to verify that the result aligns with expectations.

\section{Evaluation}

The original research conducted evaluations on different datasets to ensure the accuracy of the tool. One of these datasets is the DBPedia datasets used in the work of Wang et al.\cite{schemamanagement}, which serves as a comparison. In our case, the evaluation will be performed on the same datasets from DBPedia to assess the package’s ability to replicate. These datasets contain several JSON files, with each file containing a list of JSON objects. The evaluation aims to count the number of extracted raw schemas with ordered and unordered structures.

Upon executing the aforementioned script within the container, the experimental results are shown in \autoref{tab:results}. For each dataset, we consistently obtain the same result, confirming the system’s functional consistency.

\begin{table}[hb]% h asks to places the floating element [h]ere.
	\centering
	\caption{Comparison with Wang et al.}
	\label{tab:results}
	\begin{tabular}{|c|c|c|c|c|}
		\hline
		\multicolumn{2}{|c|}{Datasets} & \multicolumn{2}{c|}{JSON Schema Discovery} & Wang et al.                 \\
		\hline
		Collection                     & N\_JSON                                    & RS          & ROrd  & FS    \\
		\hline
		drugs                          & 3662                                       & 2818        & 2818  & 2818  \\
		\hline
		companies                      & 24367                                      & 21312       & 21312 & 21302 \\
		\hline
		movies                         & 30330                                      & 25140       & 25140 & 25137 \\
		\hline
	\end{tabular}
	\parbox{0.4\textwidth}{
		\raggedright\footnotesize
		N\_JSON - Number of JSON documents. RS - Raw schemas \newline
		ROrd - Raw schemas with ordered structure. FS - Final schemas.
	}
\end{table}

%\clearpage

\bibliographystyle{ACM-Reference-Format}
\bibliography{main}

\end{document}
\endinput
